\documentclass[repertorio-solutions-1.tex]{subfiles}
% TeorNum 115

\begin{document}

\begin{problem}{TeorNum 115}
Para todo par $(p,a)$, $p$ primo e $a$ inteiro, existem inteiros $x,y$
tais que $p|x^2+y^2+a$.
\end{problem}

\begin{solution}
Se $p=2$ então $y=0$, e $x=0$ ou $x=1$ conforme $a\equiv_2 0$ ou $a\equiv_2 1$
respetivamente, functiona para cada $a$. Agora seja $p>2$.
Se $p|a$ podemos escolher $x=y=0$.
Resta-nos agora $p\coprime a$ e $p>2$.
Consideremos o sistema de duas congruências

\begin{equation*}
n\equiv_4 1 \qquad n\equiv_p -a
\end{equation*}

As soluções deste sistema são os termos da progressão aritmética
$n\in\{4pm+k\}_{m\in\mathbb{Z}}$, para o único inteiro
$k\in\{-a,p-a,2p-a,3p-a\}$ que satisfaz $k\equiv_4 1$.
Este $k$ é coprimo com $p$ pois $-a\coprime p$, e como é ímpar
então é coprimo com $4p$.
Logo pelo Teorema de Dirichlet existem infinitos primos $n$
nesta progressão aritmética, que resolvem o sistema de congruências.
Seja então $n$ um desses primos.
Olhando para a primeira congruência, $n\equiv_4 1$, por Fermat-Euler
podemos escrever $n=x^2+y^2$ para certos inteiros $x,y$.
Isto implica, na segunda congruência, $x^2+y^2=n\equiv_p -a$,
que é o procurado $p|x^2+y^2+a$.
\end{solution}

\end{document}