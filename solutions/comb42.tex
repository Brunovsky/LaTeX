\documentclass[repertorio-solutions-1.tex]{subfiles}
% Comb 42

\newcommand*{\sets}{\{0,1,2\}}
\newcommand*{\setn}{\{1,2,\dots,n\}}

\begin{document}

\begin{problem}{Comb 42}
Seja $N(n)$ o número das aplicações
$f:\setn\to\sets$
que satisfazem $\sum_{i=1}^n f(i)=n$.
Então
$$N(n)=\sum_{k=0}^n\frac{n^{2k}}{k!^2}$$
\end{problem}

\begin{solution}
Seja $Q_n=\{f:\setn\to\sets\suchthat\sum_{i=1}^n f(i)=n\}$,
de modo que $N(n)=|Q_n|$.

Para $f:\setn\to\sets$
qualquer e cada $k=0,1,2$, seja $c_k(f)=|f^{-1}(k)|$
o número de $i\in\setn$ tais que $f(i)=k$.
Então $n=c_0(f)+c_1(f)+c_2(f)$ e $\sum_{i=1}^n f(i)=
c_1(f)+2c_2(f)$, para qualquer tal $f$.

A condição necessária e suficiente para $f\in Q_n$ é
$n=\sum_{i=1}^n f(i)$, logo
\begin{equation*}
f\in Q_n
\Leftrightarrow c_0(f)+c_1(f)+c_2(f)=c_1(f)+c_2(f)
\Leftrightarrow c_0(f)=c_2(f)
\end{equation*}
Por outras palavras, $f\in Q_n$ see houver tantos
$i\in\setn$ tais que $f(i)=2$ como $i$
tais que $f(i)=0$.

Dada $f\in Q_n$, coloramos o conjunto $\setn$
do seguinte modo: para $i\in\setn$, $i$ é
vermelho se $f(i)=0$ ou azul se $f(i)=2$. Nestas
condições, contar quantas são as funções $f\in Q_n$
é equivalente a contar de quantas formas podemos, para
cada $k\leq n$, escolher $k$ números em $\setn$
para colorir de vermelho e outros $k$ para colorir de azul. Contemos então: para cada $k\leq n$ podemos escolher os $k$ números que vamos colorir de vermelho de entre os $n$ números $\setn$ de $\binom{n}{k}$ maneiras. Os $k$ números que vamos colorir de azul escolhemo-los de entre so restantes $n-k$ números de $\binom{n-k}{k}$ maneiras. Logo no total há $\sum_k\binom{n}{k}\binom{n-k}{k}$ colorações distintas, ou seja, $N(n)=\sum_k\binom{n}{k}\binom{n-k}{k}$.

Finalmente, expandimos os termos no somatório e concluímos o problema:
\begin{equation*}
\sum_{k=0}^n\frac{n^{2k}}{k!^2}
=\sum_k\frac{n^k}{k!}\frac{(n-k)^k}{k!}
=\sum_k\binom{n}{k}\binom{n-k}{k}.\qedhere
\end{equation*}
\end{solution}

\end{document}