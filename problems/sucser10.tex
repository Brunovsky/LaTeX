\documentclass[main.tex]{subfiles}
% SucSer 10

\begin{document}

\begin{problem}{SucSer 10}
Seja $m\in\mathbb{Z}_{\geq 0}$. Então para cada $l\in\mathbb{Z}_{\geq 0}$,
existe um $p\in\mathbb{Z}_{\geq 0}$ tal que
\begin{equation*}
(\sqrt{m+1}-\sqrt{m})^l=\sqrt{p+1}-\sqrt{p}
\end{equation*}
\end{problem}

\begin{solution}
Para $m=0$ tem-se $p=0$.
Agora seja $m>0$.
Observe-se que, para quaisquer inteiros positivos $m$ e $n$
(aliás quaisquer reais não negativos):
\begin{align*}
(\sqrt{m+1}-\sqrt{m})(\sqrt{n+1}-\sqrt{n})
=(\sqrt{(m+1)(n+1)}+\sqrt{mn})-(\sqrt{(m+1)n}+\sqrt{m(n+1)}) \\
=\sqrt{2mn+m+n+2\sqrt{m(m+1)n(n+1)}+1}-\sqrt{2mn+m+n+2\sqrt{m(m+1)n(n+1)}}
\end{align*}

Este cômputo permite-nos descobrir o $p=p_l\in\mathbb{R}^+$
do enunciado recursivamente.
Para $l=1$ funciona $p_1=m$ no lugar de $p$.
Pondo $n=p_1=m$ no cômputo, o lado esquerdo fica igual a
$(\sqrt{m+1}-\sqrt{m})^2$ e o lado direito diz
$p_2=2m^2+m+m+2\sqrt{m(m+1)m(m+1)}$, etc.
Agora suponhamos que $(\sqrt{m+1}-\sqrt{m})^k=\sqrt{p_k+1}-\sqrt{p_k}$
para certos $p_k\in\mathbb{R}^+, k=1,2,...l$.
Pondo $n=p_l$ no cômputo, o lado esquerdo fica igual a
$(\sqrt{m+1}-\sqrt{m})^{l+1}$, e olhando para o lado direito
deduzimos a fórmula de recorrência para a sequência $\{p_l\}_{l\geq 1}$:
\begin{align*}
p_{l+1}=2mp_l+m+p_l+2\sqrt{m(m+1)p_l(p_l+1)} \\
(\sqrt{m+1}-\sqrt{m})^l=\sqrt{p_l+1}-\sqrt{p_l},\qquad \forall l\geq 1.
\end{align*}

O enunciado pede para provar que $\{p_l\}_{l\geq 1}$
é uma sucessão de inteiros positivos.

Provamos por indução em $l$ as duas seguintes afirmações:

$\rightarrow$  $p_l$ é inteiro (positivo).

$\rightarrow$  $p_l(p_l+1)=S^2m(m+1)$, para algum inteiro positivo $S$.

Vejamos: $p_1=m$ é inteiro, e $p_1(p_1+1)=m(m+1)$. Logo $S=1$ para $l=1$.
Isto dá o caso-base da indução.

Agora suponhamos que $l\geq 1$, $p_l$ é inteiro e $p_l(p_l+1)=A^2m(m+1)$
para algum inteiro positivo $A$.
Temos de provar que $p_{l+1}$ é inteiro e $p_{l+1}(p_{l+1}+1)=S^2m(m+1)$
para um certo $S>0$.

Pela fórmula de recorrência $p_{l+1}=2mp_l+m+p_l+2Am(m+1)$, um inteiro. E agora calculamos $p_{l+1}(p_{l+1}+1)$, mãos à obra e fé em Deus.

\begin{align*}
p_{l+1}(p_{l+1}+1)
&=(2mp_l+m+p_l+2Am(m+1))(2mp_l+m+p_l+2Am(m+1)+1) \\
&=(2mp_l+m+p_l+2Am(m+1))^2+(2mp_l+m+p_l+2Am(m+1)) \\
&=(4m^2p_l^2+4m^2p_l+4mp_l^2+m^2+p_l^2+2mp_l+4A^2m^2(m+1)^2 \\
&\quad +4Am(m+1)(2mp_l+m+p_l)) + (2mp_l+m+p_l+2Am(m+1)) \\
&=(4m^2p_l^2+4m^2p_l+4mp_l^2+4mp_l+m^2+m+p_l^2+p_l \\
&\quad +4A^2m^2(m+1)^2+4Am(m+1)(2mp_l+m+p_l)) + 2Am(m+1) \\
&=4m(m+1)p_l(p_l+1)+m(m+1)+p_l(p_l+1) \\
&\quad 4A^2m^2(m+1)^2+4Am(m+1)(2mp_l+m+p_l) + 2Am(m+1) \\
&=4m(m+1)p_l(p_l+1)+m(m+1)+A^2m(m+1) \\
&\quad 4A^2m^2(m+1)^2+4Am(m+1)(2mp_l+m+p_l) + 2Am(m+1) \\
&=m(m+1)\Big(4p_l(p_l+1)+1+A^2+4A^2m(m+1)+4A(2mp_l+m+p_l)+2A\Big) \\
&=m(m+1)\Big(4p_l^2+4p_l+(A+1)^2+4A^2m^2+4A^2m+8Amp_l+4Am+4Ap_l\Big) \\
&=m(m+1)\Big((A+1)^2+4(A^2m^2+2Amp_l+p_l^2)+4(p_l+A^2m+Am+Ap_l)\Big) \\
&=m(m+1)\Big((A+1)^2+4(Am+p_l)^2+4(A+1)(Am+p_l)\Big) \\
&=m(m+1)\big(A+1+2Am+2p_l\big)^2.
\end{align*}
E por fim $S=A+1+2Am+2p_l\in\mathbb{Z}_{> 0}$.
Isto termina o passo de indução e o problema.
\end{solution}

\end{document}