\documentclass[main.tex]{subfiles}
% TeorNum 103

\renewcommand*{\ord}[1]{\ensuremath{\text{ord}_#1\:}}

\begin{document}

\begin{problem}{TeorNum 103}
Quais os pares de primos $(p,q)$ tais que $pq|p^p+q^q+1$?
\end{problem}

\begin{solution}
\begin{note}
Recordemos que, se $a\in\mathbb{Z}_m^\times$ - conjunto dos inteiros $\leq m$ coprimos com $m$ - e a ordem multiplicativa de $a$ módulo $m$ é $d$ (i.e. $d$ é o menor natural tal que $a^d\equiv_m 1$), então $d$ divide todo natural $k$ tal que $a^k\equiv_m 1$.
Uma demonstração simples: se $k=nd+r$ com $0\leq r<d$ então
$a^k\equiv_m a^{nd+r}\equiv_ma^{nd}a^r\equiv_ma^r$. Para não se contradizer $d$ ser o menor natural tal que $a^d\equiv_m 1$, então $a^k\equiv_m 1$ sse $r=$, i.e. sse $d|k$.
\end{note}

Os primos $p$ e $q$ podem ser permutados. Suponhamos por agora que um dos primos $p$ e $q$ é igual a $2$, por exemplo $p=2$. A divisibilidade fica simplesmente $2q|q^q+5$. Em particular $q|2q|q^q+5$ logo $q|5\Rightarrow q=5$.
Por outro lado os pares $(p,q)=(2,5),(5,2)$ satisfazem o enunciado, pois
$p^p+q^q+1=2^2+5^5+1\equiv 4+5+1\equiv 0\pmod{pq=10}$.

Agora suponhamos que $p$ e $q$ são primos ímpares tais que $pq|p^p+q^q+1$.
$p=q$ claramente não funciona, pois conduz a $p^2=pq|p^p+q^q+1=2p^p+1\Rightarrow p^2|1$, um absurdo.
Logo $p\coprime q$, e a divisibilidade $pq|p^p+q^q+1$ é equivalente às duas divisibilidades, simultâneas, seguintes:
$p|p^p+q^q+1$ e $q|p^p+q^q+1$, ou seja,

\begin{equation*}
p|q^q+1 \qquad q|p^p+1
\end{equation*}

Podemos então escrever
\begin{align}
q^q&\equiv_p -1 & p^p&\equiv_q -1\\
q^{2q}&\equiv_p +1 & p^{2p}&\equiv_q +1
\end{align}

Notemos que $1\not\equiv -1$ módulo $p$ ou $q$ pois estes são primos ímpares.

Seja $d_q=\ord{p} q$ a ordem multiplicativa de $q$ módulo $p$,
e $d_p=\ord{q} p$ a ordem multiplicativa de $p$ módulo $q$.
Então, de acordo com $(2)$ e com as considerações iniciais, $d_q|2q$ e $d_p|2p$.

Por pequeno Fermat $q^{p-1}\equiv_p 1$ e $p^{q-1}\equiv_q 1$,
logo $d_q|p-1$ e $d_p|q-1$.

Juntando as duas divisibilidades temos $d_q|\mdc(2q,p-1)$ e $d_p|\mdc(2p,q-1)$, ou seja,

\begin{equation}
d_q|2\mdc(q,p-1)\qquad d_p|2\mdc(p,q-1)
\end{equation}

Note-se que $\mdc(q,p-1)\in\{1,q\}$ e $\mdc(p,q-1)\in\{1,p\}$.
Suponhamos que temos $\mdc(q,p-1)=q$ e $\mdc(p,q-1)=p$.
Então $q|p-1$ e $p|q-1$, logo, respetivamente, $p-1\geq q$ e $q-1\geq p$.
Isto implica $p-1\geq q\geq p+1$, que é absurdo.

Logo pelo menos um dos mdc referidos é 1, digamos, spdg por simetria, $\mdc(q,p-1)=1$.
Então $(3)$ diz-nos que $d_q|2$.
Se fosse $d_q=1$ (i.e. $q^1\equiv_p 1$ então teríamos $q\equiv_p 1\Rightarrow q^q\equiv_p 1$, uma contradiçao com a congruência $(1)$.
Logo $d_q=2$ e $q\equiv_p -1\Rightarrow p|q+1$.

Agora $p>2$ implica $p\nmid q-1$ logo $\mdc(p,q-1)=1$ e $(3)$ diz-nos que $d_q|2$. De forma análoga $d_p=1$ leva a $p\equiv_q 1\Rightarrow p^p\equiv_q 1$, contradição com $(1)$. Logo $d_p=2$ e $p\equiv_q -1\Rightarrow q|p+1$.

Em suma temos $p|q+1$ e $q|p+1$, logo, respetivamente, $q+1\geq p$ e $p+1\geq q$.
Juntando as duas desigualdades, $p+1\geq q\geq p-1$.
Logo $q\in\{p-1,p,p+1\}$.
Mas $p\neq q$ e $p-1,p+1$ são pares, e portanto não são primos ímpares. Então chegámos a um absurdo. Logo não há mais soluções.

$\therefore$ As únicas soluções são $(p,q)\in\{(2,5),(5,2)\}$.
\end{solution}

\end{document}