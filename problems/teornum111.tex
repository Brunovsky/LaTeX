\documentclass[main.tex]{subfiles}
% TeorNum 111

\renewcommand*{\ord}[1]{\ensuremath{\text{ord}_#1\:}}

\begin{document}

\begin{problem}{TeorNum 111}
Seja $p$ primo. Então $p$ é primo de Fermat sse todo o elemento
em $\mathbb{Z}_p^*$ é quadrado módulo $p$ ou raíz primitiva módulo $p$.
\end{problem}

\begin{solution}
\begin{smallnote}
Um primo de Fermat é um primo da forma $p=2^{2^n}+1$ onde $n\in\mathbb{N}_0$.
Equivalente definição é: um primo de Fermat é um primo da forma $p=2^k+1$ onde $k\in\mathbb{N}$.
Mas se $k$ for uma potência de $2$ então $p$ é certamente composto. Para ver isto escreva-se $k=2^na$ para $n\geq 0$ e $a$ ímpar.
Então temos a fatorização
$p=2^{2^na}+1=(2^{2^n}+1)((2^{2^n})^{a-1}-(2^{2^n})^{a-2}+\dots-2^{2^n}+1)$.
Se $a>1$ então o termo na direita deste produto é $>1$ e portanto $p$ é composto.
Assim se $p$ é primo então $a=1$, e $k=2^n$ é uma potência de $2$. Notemos que $p$ é primo de Fermat sse todo o divisor de $p-1$ é par.
\end{smallnote}

Seja $g$ uma raíz primitiva módulo $p$.

Consideremos as $\frac{p-1}{2}$ potências pares
$g^{2k}\in\{g^0,g^2,\dots,g^{p-3}\}$ de $g$.
Estas potências são todas distintas módulo $p$ pois $g$ é raíz primitiva,
e são todas claramente resíduos quadráticos módulo $p$.
Pela teoria da reciprocidade quadrática sabemos que $\mathbb{Z}_p^*$ tem
exatamente $\frac{p-1}{2}$ resíduos quadráticos, logo estes são todos os
resíduos quadráticos de $\mathbb{Z}_p^*$. Por outro lado,
$(g^{2k})^{(p-1)/2)}\equiv_p (g^{p-1})^k\equiv_p 1$, logo estas potências
$g^{2k}$ não são raízes primitivas, qualquer que seja o primo $p$.

Logo as restantes $\frac{p-1}{2}$ potências ímpares
$g^k\in\{g^1,g^3,\dots,g^{p-2}\}$ são os não resíduos quadráticos módulo $p$.
Queremos mostrar que estes não resíduos quadráticos são todos raízes
primitivas módulo $p$ sse $p$ é primo de Fermat.
\\

Suponhamos por agora que $p$ é primo de Fermat.
Suponhamos para absurdo que existe um $k\in\{1,3,5,\dots,p-2\}$ ímpar para o
qual $g^k$ não é raíz primitiva módulo $p$, i.e. a ordem $d=\ord{p}g^k<p-1$.
Então $g^{kd}\equiv_p(g^k)^d\equiv_p 1$.
Como a ordem de $g$ é $p-1$ concluímos que $p-1|kd$.
Como assumimos $d<p-1$, então $p-1\nmid d$, e portanto $k$ e $p-1$ não
podem ser coprimos, i.e. $k$ e $p-1$ têm um fator em comum.
Como $p-1=2^{2^n}$ é uma potência de 2, então $k$ é par, contradição.
Logo os $\frac{p-1}{2}$ não resíduos $g^k\in\{g^1,g^3,\dots,g^{p-2}\}$
são todos raízes primitivas módulo $p$.
Isto prova uma das implicações: se $p$ é primo de Fermat então todo o
elemento em $\mathbb{Z}_p^*$ é ou resíduo quadrático módulo $p$ ou raíz
primitiva módulo $p$.
\\

Agora suponhamos que $p$ não é primo de Fermat.
Então $p-1$ tem um divisor ímpar superior a $1$, digamos $d$.
Então $g^d\in\{g^1,g^3,\dots,g^{p-2}\}$ é um resíduo não quadrático.
Por outro lado, $(g^d)^{(p-1)/d}=g^{p-1}\equiv_p 1$, logo
$\ord{p}g^k=\frac{p-1}{d}<p-1$, ou seja, $g^d$ não é raíz primitiva
módulo $p$.
Então encontrámos um elemento em $\mathbb{Z}_p^*$ que não é quadrado
módulo $p$ nem raíz primitiva módulo $p$.
Logo se $p$ não é primo de Fermat, $\mathbb{Z}_p^*$ tem elementos que
não são resíduo quadrático módulo $p$ nem raíz primitiva módulo $p$. 
\end{solution}

\end{document}