\documentclass[environments-demo.tex]{subfiles}

\newcommand{\getlength}[2]{
    \newlength\abc
    #1\settowidth\abc{#2}
    \the\abc\relax
}


\begin{document}
\section{center, flushleft, flushright, verbatim, quotation, quote, tabbing, verse}
A regular paragraph
(first in the section, therefore not indented).
\blindtext

Another regular paragraph
(not first in the section, therefore indented).
\blindtext

\begin{center}
> Now a paragraph inside a center environment.
\blindtext

\blindtext
\end{center}
\begin{flushleft}
> Now a paragraph inside a flushleft environment.
Notice the space before and after the environment.
The center environment automatically created a new paragraph.
\blindtext
\end{flushleft}

\begin{flushright}
> Now a paragraph inside a flushright environment.
The flush environments do not automatically create new paragraphs. They also do not "indent" the first line.
\blindtext
\end{flushright}

A \verb+VERB+ written with the \verb+\verb+ command
looks just like text written inside a verbatim environment
using a typewriter:

\begin{verbatim}
    Inside the verbatim environment I can write
               whatever I want,
                               wherever I want!
    Even commands \renewcommand{\documentclass}{} are ignored. \xesde
\end{verbatim}

\begin{quotation}
A quotation environment.
\blindtext
\end{quotation}

\begin{quote}
A quote environment.
\blindtext
\end{quote}

The following is a tabbing environment. It is pretty nasty and complicated,
but extremely flexible and useful for creating new environments.

\begin{tabbing}
texttext 1,1 \= texttext 1,2 \= texttext 1,3 \= texttext 1,4 \= texttext 1,5 \= texttext 1,6 \kill
text 2,1 \> text 2,2 \> text 2,3 \> text 2,4 \+\\
text 3,1 \> text 3,2 \> text 3,3 \> text 3,4 \-\\
text 4,1 \> text 4,2 \> text 4,3 \> text 4,4 \+\+\\
text \' 5,1 \> text 5,2 \> text 5,3 \> text 5,4 \-\\
text 6,1 \> text 6,2 \> text 6,3 \> text 6,4 \-\\
text 7,1 \> text 7,2 \> text 7,3 \` text 7,4 \\
text 8,1 \> text 8,2 \> text 8,3 \> text 8,4 \> text 8,5 \+\+\\
\< hello \> text n,1 \> text n,2 \> text n,3 \> text n,4 %\\
\end{tabbing}

\begin{verse}
O amor, quando se revela,\\
Não se sabe revelar.\\
Sabe bem olhar p'ra ela,\\
Mas não lhe sabe falar.

Quem quer dizer o que sente\\
Não sabe o que há de dizer.\\
Fala: parece que mente...\\
Cala: parece esquecer...

Ah, mas se ela adivinhasse,\\
Se pudesse ouvir o olhar,\\
E se um olhar lhe bastasse\\
P'ra saber que a estão a amar!

Mas quem sente muito, cala;\\
Quem quer dizer quanto sente\\
Fica sem alma nem fala,\\
Fica só, inteiramente!

Mas se isto puder contar-lhe\\
O que não lhe ouso contar,\\
Já não terei que falar-lhe\\
Porque lhe estou a falar...

Fernando Pessoa
\end{verse}

\begin{center}
\emph{Emphasized and \emph{not emphasized} text}

\textit{Italicized and \textit{still italicized} text}

\textbf{Bold and \textbf{still bold} text}

\textsc{Small capitals and \textsc{still small capitals} text}

\uline{Underlined and \uline{double underlined} text}

\end{center}

\getlength{\tiny
}{11}

\end{document}